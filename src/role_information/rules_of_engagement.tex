% =====================================================
% TAS Rules of Engagement
% =====================================================

\clearpage
\phantomsection
\section{TAS Rules of Engagement}
\label{sec:tas-rules-of-engagement}

\begin{center}
    {\Large \textbf{Technical Alignment Superhero (TAS)}}\\[6pt]
    {\Large \textbf{Rules of Engagement}}\\[4pt]
    {\small (Governing assessment behavior, communication, and escalation)}
\end{center}

\vspace{0.6cm}

% =====================================================
\subsection{Purpose and Scope}

\textbf{Purpose}
\begin{itemize}
    \item Establish clear operational boundaries for the TAS role
    \item Ensure consistency and professionalism during alignment engagements
    \item Protect the integrity of myITprocess alignment outputs
    \item Maintain separation between technical assessment and business strategy
\end{itemize}

\textbf{Scope}
\begin{itemize}
    \item Applies to all TAS-led alignment activities:
    \begin{itemize}
        \item Pre-visit preparation
        \item Onsite validation
        \item Post-visit documentation
        \item vCIO handoff
    \end{itemize}
    \item Applies to all client-facing and internal interactions during alignment work
\end{itemize}

% =====================================================
\subsection{TAS Role Definition (Operational View)}

\textbf{The TAS \emph{IS} responsible for:}
\begin{itemize}
    \item Validating the factual technical state of client environments
    \item Collecting evidence to support alignment standards
    \item Updating and correcting documentation (ITGlue)
    \item Completing alignment reviews in myITprocess
    \item Identifying gaps, risks, and unknowns
    \item Escalating concerns to the vCIO
\end{itemize}

\textbf{The TAS \emph{IS NOT} responsible for:}
\begin{itemize}
    \item Finalizing or presenting strategic recommendations
    \item Prioritizing business risk or impact
    \item Quoting costs, timelines, or project scope
    \item Approving remediation actions
    \item Redefining client strategy during assessment
\end{itemize}

% =====================================================
\subsection{ Engagement Boundaries}

\subsubsection{Assessment vs.\ Remediation}

\textbf{Assessment Phase Rules}
\begin{itemize}
    \item TAS may observe, validate, and document
    \item TAS must not perform remediation unless explicitly authorized
    \item All remediation needs must be documented via myITprocess tickets
\end{itemize}

\textbf{Remediation Phase Rules}
\begin{itemize}
    \item Remediation is performed by Support or Projects teams
    \item TAS involvement is limited to validation and confirmation
\end{itemize}

\subsubsection{Communication Boundaries with Clients}

\textbf{TAS may say:}
\begin{itemize}
    \item ``Here is what I observed.''
    \item ``This appears aligned / not aligned to the standard.''
    \item ``I will document this for review by your vCIO.''
    \item ``We will follow up with recommendations after review.''
\end{itemize}

\textbf{TAS must not say:}
\begin{itemize}
    \item ``You should do X.''
    \item ``This will cost about Y.''
    \item ``This is critical and needs to be fixed now.''
    \item ``I recommend \ldots''
\end{itemize}

% =====================================================
\subsection{Evidence and Documentation Standards}

\textbf{All findings must be:}
\begin{itemize}
    \item Fact-based
    \item Verifiable
    \item Supported by evidence where possible
\end{itemize}

\textbf{Acceptable evidence includes:}
\begin{itemize}
    \item Screenshots
    \item Configuration exports
    \item Tool dashboards
    \item Physical inspection notes
    \item Stakeholder statements (clearly labeled)
\end{itemize}

\textbf{Evidence rules}
\begin{itemize}
    \item Opinion is not evidence
    \item Assumptions must be explicitly labeled
    \item Unknowns must remain Unknown until validated
\end{itemize}

% =====================================================
\subsection{Handling Client Pushback or Pressure}

\textbf{If a client requests recommendations onsite:}
\begin{itemize}
    \item Acknowledge the concern
    \item Explain the review and approval process
    \item Defer recommendations to vCIO review
\end{itemize}

\textbf{If a client requests immediate remediation:}
\begin{itemize}
    \item Explain the separation of assessment and remediation
    \item Create a myITprocess ticket if appropriate
    \item Escalate urgent issues internally
\end{itemize}

\textbf{If a client disputes a finding:}
\begin{itemize}
    \item Re-validate facts where possible
    \item Capture the client’s statement verbatim
    \item Do not remove findings without evidence
\end{itemize}

% =====================================================
\subsection{Stakeholder Interaction Guidelines}

\textbf{During interviews and discussions:}
\begin{itemize}
    \item Listen more than you speak
    \item Capture exact language when possible
    \item Avoid validating emotions with technical conclusions
    \item Separate business pain from technical cause
\end{itemize}

\textbf{Stakeholder statements must be:}
\begin{itemize}
    \item Attributed (role/title)
    \item Clearly marked as statements, not facts
    \item Included for business context only
\end{itemize}

% =====================================================
\subsection{Risk Identification and Escalation}

\textbf{TAS responsibilities include:}
\begin{itemize}
    \item Flagging high-risk technical findings
    \item Identifying single points of failure
    \item Highlighting compliance or security red flags
\end{itemize}

\textbf{TAS is not responsible for:}
\begin{itemize}
    \item Assigning business severity
    \item Determining urgency or priority
    \item Deciding remediation sequencing
\end{itemize}

\textbf{Urgent issues}
\begin{itemize}
    \item Must be escalated internally the same day
    \item Must still be documented properly
\end{itemize}

% =====================================================
\subsection{myITprocess Usage Rules}

\textbf{Alignment reviews}
\begin{itemize}
    \item All applicable standards must be answered
    \item ``Unknown'' responses must include reason and next step
    \item Notes must be understandable without TAS presence
\end{itemize}

\textbf{Tickets}
\begin{itemize}
    \item Created in myITprocess when remediation is required
    \item Must include:
    \begin{itemize}
        \item Supporting evidence
        \item Impacted systems
        \item Summary of issue
    \end{itemize}
\end{itemize}

% =====================================================
\subsection{Professional Conduct and Ethics}

\textbf{TAS must:}
\begin{itemize}
    \item Remain neutral and factual
    \item Avoid blaming prior providers or client staff
    \item Avoid overpromising outcomes
    \item Maintain confidentiality
\end{itemize}

\textbf{TAS must not:}
\begin{itemize}
    \item Undermine client confidence
    \item Speak negatively about colleagues or vendors
    \item Provide unofficial guidance
\end{itemize}

% =====================================================
\subsection{Exceptions and Escalation}

\textbf{Exceptions}
\begin{itemize}
    \item Require explicit leadership approval
    \item Must be documented
\end{itemize}

\textbf{When in doubt}
\begin{itemize}
    \item Pause
    \item Document
    \item Escalate to vCIO or leadership
\end{itemize}

% =====================================================
\subsection{Enforcement and Review}

\textbf{Violations of these rules may result in:}
\begin{itemize}
    \item Coaching
    \item Revision of findings
    \item Removal from alignment duties for repeated violations
\end{itemize}

\textbf{Review cadence}
\begin{itemize}
    \item Annually
    \item When myITprocess standards change
    \item When TAS role scope evolves
\end{itemize}

