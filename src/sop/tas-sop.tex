% src/sop/tas-sop.tex
% NOTE: This file is intended to be INCLUDED by playbook.tex via \include{}
% Do NOT add \documentclass, \usepackage, \begin{document}, \end{document}, or \pagestyle here.

% ------------------------------------------------------------
% SOP Title (playbook-friendly)
% ------------------------------------------------------------
\clearpage
\section*{Technical Alignment Superhero (TAS) Standard Operating Procedures}
\addcontentsline{toc}{section}{Technical Alignment Superhero (TAS) Standard Operating Procedures}

\noindent
\textbf{Purpose:} Define the standard operating procedures governing the Technical Alignment Superhero (TAS) role at Tekie Geek. This SOP is the governing backbone for alignment reviews and all related TAS playbook artifacts.\\[6pt]
\textbf{Scope:} Applies to all TAS-led alignment reviews, including preparation, onsite validation, documentation, reporting, and follow-up.\\[6pt]
\textbf{Authority:} TAS documents facts and alignment observations. vCIO approves risk scoring, prioritization, and client-facing recommendations.

\vspace{0.75cm}

% ------------------------------------------------------------
% Step 1
% ------------------------------------------------------------
\clearpage
\section{Pre-Assessment Prep Work}

\textbf{Goal:} Establish the framework, standards, and tools required to perform a consistent and objective technical alignment review. Ensure readiness prior to engaging in any client-specific evaluation. No client-specific findings or alignment gaps should be identified during this step.

\subsection{Define Alignment Standards}

Prior to reviewing any client environment, alignment standards must be defined and approved in myITprocess. These standards will be used later to perform a formal alignment assessment once accurate and complete information has been validated.

Each standard \textbf{must be phrased as a yes/no question} with no room for ambiguity.

\textbf{Rule:} If a standard cannot be definitively answered with ``Yes'' or ``No,'' it must be rewritten.

\subsubsection{Technology Standards (Examples)}

\begin{enumerate}
    \item Are all endpoints running a supported OS version approved by Tekie Geek?
    \item Are all endpoints fully patched with the latest OS and security updates?
    \item Do all endpoints have the full security stack deployed based on the client’s current contract (SentinelOne, Huntress, ThreatLocker, SaaS Alerts)?
    \item Are all servers and required endpoints included in a monitored and verified backup solution (Datto BCDR, Datto Backup for PCs)?
    \item Are all firewalls:
    \begin{itemize}
        \item Supported by the vendor (ControlOne)?
        \item Actively licensed?
        \item Managed and monitored by Tekie Geek?
    \end{itemize}
\end{enumerate}

\subsection{Define Lifecycle Standards}

Lifecycle standards must be defined prior to assessment to ensure consistent and objective evaluation of client hardware and infrastructure.

\begin{itemize}
    \item \textbf{Workstations:} 3--5 years
    \item \textbf{Servers:} 5 years
    \item \textbf{Firewalls:} 3--5 years
\end{itemize}

Lifecycle standards should be applied consistently across all clients unless otherwise documented due to regulatory, operational, or business-specific requirements.

\subsection{Finalize Alignment Tool Structure}

\subsubsection{Categories}

Define the core alignment categories to be used during assessment:

\begin{itemize}
    \item Infrastructure
    \item Security
    \item Backup and Disaster Recovery
    \item Endpoint Management
    \item Network and Connectivity
    \item Documentation
\end{itemize}

\subsubsection{Risk Scoring}

Define a standardized risk scoring methodology that is simple, consistent, and easily understood by both technical and non-technical stakeholders. Note: TAS collects observations and initial scoring of yes or no (aligned or not aligned); final risk scoring is approved by the vCIO.

\subsubsection{3.3.3 Notes Format}

All sections of the technical alignment review \textbf{must include structured notes} that clearly document the factual basis for each alignment decision. Notes serve as the authoritative record supporting \textit{Yes / No / Unknown} responses and must be understandable without additional verbal explanation.

\paragraph{Required Note Components}

Each alignment standard note \textbf{must include the following elements}, in order:

\begin{enumerate}
    \item \textbf{Observed State}
    \begin{itemize}
        \item A concise, factual description of what was observed.
        \item Must describe the current technical condition, configuration, or behavior.
        \item Avoid conclusions, opinions, or recommendations.
    \end{itemize}

    \item \textbf{Evidence Source}
    \begin{itemize}
        \item Identify how the observation was validated.
        \item Acceptable sources include:
        \begin{itemize}
            \item Tool name and location (e.g., Kaseya VSA device view)
            \item Screenshot or photo reference
            \item Configuration export
            \item Documentation reference (ITGlue object name)
            \item Stakeholder statement (clearly labeled)
        \end{itemize}
    \end{itemize}

    \item \textbf{Alignment Determination}
    \begin{itemize}
        \item Explicitly state why the standard was marked:
        \begin{itemize}
            \item \textbf{Yes} — Fully meets the standard
            \item \textbf{No} — Does not meet the standard
            \item \textbf{Unknown} — Insufficient information to determine
        \end{itemize}
        \item Tie the determination directly to the observed state and evidence.
    \end{itemize}

    \item \textbf{Scope or Impact Context (if applicable)}
    \begin{itemize}
        \item Clarify how widespread the issue is.
        \item Examples include:
        \begin{itemize}
            \item ``Affects 3 of 42 endpoints''
            \item ``Applies only to the primary site''
            \item ``Limited to legacy systems''
        \end{itemize}
        \item Do \textbf{not} assign business priority or severity.
    \end{itemize}

    \item \textbf{Next Step Indicator (for No or Unknown only)}
    \begin{itemize}
        \item Indicate what is required to resolve uncertainty or address the gap.
        \item Examples include:
        \begin{itemize}
            \item ``Requires onsite validation''
            \item ``Requires configuration remediation''
            \item ``Requires vCIO review for business context''
        \end{itemize}
    \end{itemize}
\end{enumerate}

\paragraph{Formatting Rules}

\begin{itemize}
    \item Notes must be:
    \begin{itemize}
        \item Clear and concise
        \item Written in complete sentences
        \item Free of speculation or emotional language
    \end{itemize}
    \item Notes must \textbf{not} include:
    \begin{itemize}
        \item Cost estimates
        \item Timelines
        \item Recommendations phrased as directives
    \end{itemize}
    \item Stakeholder input must be clearly labeled as:
    \begin{itemize}
        \item \textit{Client-stated} or \textit{Stakeholder-reported}
    \end{itemize}
\end{itemize}

\paragraph{Acceptable Language Examples}

\begin{itemize}
    \item ``Observed that 12 of 38 endpoints are missing SentinelOne agent based on RMM agent comparison.''
    \item ``Firewall configuration validated via ControlOne dashboard; license active and device online.''
    \item ``Unable to validate backup retention for NAS device; documentation missing and device unavailable onsite.''
\end{itemize}

\paragraph{Unacceptable Language Examples}

\begin{itemize}
    \item ``This is really bad and needs to be fixed.''
    \item ``Client should upgrade immediately.''
    \item ``Probably not compliant.''
    \item ``Looks okay.''
\end{itemize}

\paragraph{Purpose of Notes}

Notes exist to:
\begin{itemize}
    \item Defend alignment decisions during vCIO review
    \item Enable consistent reporting across clients
    \item Support auditability and historical tracking
    \item Reduce rework and follow-up clarification
\end{itemize}

If a third party (vCIO, auditor, or leadership) cannot understand \textbf{what was observed and why the alignment decision was made}, the note is considered incomplete.

\paragraph{Optional Internal Template}

TASs may optionally structure notes using the following internal format:

\begin{verbatim}
Observed State:
Evidence:
Alignment Decision:
Scope / Impact:
Next Step:
\end{verbatim}


\subsection{Pre-Assessment Prep Work Output}

\begin{itemize}
    \item A finalized set of alignment and lifecycle standards
    \item A finalized alignment tool structure ready for use in myITprocess
    \item Clear readiness to begin client-specific planning and scheduling
\end{itemize}

% ------------------------------------------------------------
% Step 2
% ------------------------------------------------------------
\clearpage
\section{Client Planning and Scheduling}

\textbf{Goal:} Ensure the TAS arrives onsite with appropriate context, access, stakeholders, and priorities. Plan the engagement and identify areas requiring onsite validation. All meetings must be documented in myITprocess using the scheduling feature.

\subsection{Internal Planning Checklist}

\begin{itemize}
    \item Review current documentation in ITGlue
    \item Review open risks in myITprocess
    \item Review ticket trends in BMS from the previous 30--90 days
    \item Identify areas requiring onsite validation or clarification
    \item Confirm the primary technical contact
    \item Confirm the business decision-maker (owner, Executive Director, COO, etc.)
\end{itemize}

\subsection{Client Outreach}

\begin{itemize}
    \item Introduce the TAS role and responsibilities
    \item Explain the purpose and objectives of the onsite visit
    \item Describe benefits to the client, including improved IT alignment, risk mitigation, and strategic recommendations
    \item Discuss logistics: duration, required attendees, access requirements
    \item Outline follow-up process and reporting expectations
    \item Provide a high-level agenda if appropriate
\end{itemize}

\subsubsection*{Client Outreach Email Boilerplat}

\begin{tcolorbox}[
    colback=gray!10,
    colframe=gray!70,
    boxrule=0.8pt,
    arc=4pt,
    left=6pt,
    right=6pt
]
\raggedright

\textbf{Subject:} Upcoming Alignment Review Visit

\vspace{6pt}

Hi [Client Name],

\vspace{6pt}

I’m [Your Name], the Technical Alignment Superhero (TAS) working with your team at Tekie Geek.

\vspace{6pt}

I’ll be onsite on \textbf{[Date]} to review and validate the technical side of your environment. The goal of this visit is to make sure everything we base future planning on is accurate and up to date.

\vspace{6pt}

During the visit, I’ll be working quietly in the background to validate systems and documentation and capture anything that needs follow-up with your vCIO. No recommendations or changes will be finalized onsite.

\vspace{6pt}

Before the visit, could you please confirm:
\begin{itemize}
    \item The best technical point of contact
    \item Any stakeholders who should be available
    \item Any access or timing considerations we should be aware of
\end{itemize}

\vspace{6pt}

Looking forward to working with you.

\vspace{6pt}

Best regards,\\
[Your Name]\\
Technical Alignment Superhero\\
Tekie Geek

\end{tcolorbox}


\subsection{Documentation Review (ITGlue)}

Review ITGlue to build familiarity with the client environment. Missing or incomplete documentation should be treated as a planning consideration rather than a formal alignment finding.

\subsubsection{ISP \& Connectivity Overview}

\begin{itemize}
    \item Primary ISP provider
    \item Secondary ISP provider (if applicable)
    \item Link type (fiber, coax, DSL, etc.)
    \item Account number(s)
    \item Security codes or PINs
    \item Upload and download speeds
    \item Firewall make and model
    \item Relevant public IP addresses (static or dynamic)
\end{itemize}

\subsubsection{Core Documentation Objects to Review}

\begin{itemize}
    \item Network diagram (logical/physical)
    \item Firewall documentation and configuration
    \item Server inventory (physical/virtual)
    \item Endpoint inventory
    \item Backup configuration and scope (Datto BCDR, Datto Backup for PCs)
    \item M365 / cloud tenant documentation
    \item Passportal credentials
    \item Vendor contacts and escalation paths
\end{itemize}

\subsection{Client Scheduling and Planning Output}

\begin{itemize}
    \item Clear engagement plan and onsite agenda
    \item Confirmed client stakeholders and points of contact
    \item List of items requiring onsite validation
    \item Readiness to proceed with onsite validation and discovery
\end{itemize}

% ------------------------------------------------------------
% Step 3
% ------------------------------------------------------------
\clearpage
\section{Pre-Visit Data Gathering}

\textbf{Goal:} Minimize time onsite by doing thorough homework beforehand. Ensure the TAS arrives with accurate, up-to-date information and a clear action plan. No formal alignment findings are generated during this phase.

\subsection{Pre-Visit Checklist}
Click \hyperref[sec:previsit-checklist]{here} to see the full previsit checklist for full execution steps.

\begin{itemize}
    \item \textbf{RMM Device List (Kaseya VSA):} Confirm current endpoints and server population
    \item \textbf{Backup Reports (Datto BCDR / Datto Backup for PCs / Kaseya BMS):} Analyze recent backup jobs
    \item \textbf{Security Alerts (SentinelOne, Huntress, ThreatLocker, SaaS Alerts):} Review antivirus, firewall, and endpoint alerts
    \item \textbf{M365 Tenant Review:} Examine configuration, security, license usage, and audit logs
    \item \textbf{Network Diagrams (if available):} Validate or supplement existing diagrams
    \item \textbf{Pre-Fill Alignment Tool (myITprocess):} Input known information and preliminary observations
    \item \textbf{Onsite Checklist:} Prepare physical or digital checklist for validation and documentation
\end{itemize}

\subsection{Pre-Visit Data Gathering Output}

\begin{itemize}
    \item Updated and validated inventory of monitored devices
    \item Awareness of backup health and potential risks
    \item Understanding of recent or active security issues
    \item Insight into the client’s cloud environment posture
    \item Pre-populated alignment tool ready for onsite validation
    \item Clear, actionable checklist for the onsite visit
\end{itemize}

% ------------------------------------------------------------
% Step 4
% ------------------------------------------------------------
\clearpage
\section{Onsite Validation and Discovery}

\textbf{Goal:} Validate pre-collected information, gather missing data, and identify alignment gaps for formal reporting. Ensure the client environment is accurately documented in ITGlue and myITprocess.

\subsection{Onsite Checklist}
Click \hyperref[sec:onsite-checklist]{here} to see the full onsite checklist for full execution steps.

\begin{itemize}
    \item Validate endpoint and server inventory via Kaseya VSA
    \item Confirm backup success and alert status (Datto BCDR, Datto Backup for PCs, Kaseya BMS)
    \item Validate endpoint security coverage (SentinelOne, Huntress, ThreatLocker, SaaS Alerts)
    \item Check firewall configuration and management status (ControlOne)
    \item Confirm ransomware protection status (Canauri)
    \item Verify cloud tenant configuration (M365) and license usage
    \item Update ITGlue documentation with any missing or corrected items
    \item Log onsite findings in myITprocess for traceability
\end{itemize}

\subsection{Best Practices}

\begin{itemize}
    \item Take photos or screenshots of critical systems for documentation
    \item Use tool dashboards (Kaseya VSA, Huntress, SentinelOne) to quickly validate system health
    \item Tag and prioritize high-risk technical concerns in myITprocess for vCIO review
    \item Maintain continuous communication with client contacts to clarify discrepancies
    \item If remediation is required, a ticket \textbf{must} be created in myITprocess. This will automatically generate a corresponding ticket in BMS for the IT Support team to perform remediation.
    \item TAS is responsible for capturing and validating factual data; the vCIO will prioritize, approve, and communicate any strategic recommendations
\end{itemize}

\subsection{Onsite Validation and Discovery Output}

\begin{itemize}
    \item Verified and updated documentation in ITGlue
    \item Comprehensive alignment observations logged in myITprocess
    \item Preliminary list of alignment gaps for vCIO review
    \item Alignment reviews must be completed on the same day as the onsite visit unless explicitly approved by leadership
\end{itemize}

% ------------------------------------------------------------
% Step 5
% ------------------------------------------------------------
\clearpage
\section{Reporting and Recommendations}

\textbf{Goal:} Produce an actionable alignment report summarizing observations, gaps, risks, and recommended remediations for vCIO review.

\subsection{Report Components}

\begin{itemize}
    \item Executive Summary: Key findings and technical overview
    \item Detailed Findings: Infrastructure, Security, Backup, Endpoint Management, Network Connectivity, Documentation
    \item Remediation Recommendations: Initial TAS recommendations captured; final recommendations must be reviewed and approved by the vCIO before client delivery
    \item Risk Scoring: Preliminary risk scoring done by TAS, finalized by the vCIO based on business impact and urgency
    \item Visuals: Diagrams, screenshots, and supporting evidence from ITGlue, Kaseya, Datto, and security dashboards
\end{itemize}

\subsection{Best Practices}

\begin{itemize}
    \item Use myITprocess to track remediation actions and statuses
    \item Reference screenshots or logs from SentinelOne, Huntress, Datto, and Canauri as evidence
    \item Ensure findings are understandable by both technical and executive audiences
    \item TAS must \textbf{never} present recommendations directly to the client without vCIO approval
    \item vCIO validates recommendations and may adjust priorities or approach based on business context
\end{itemize}

\subsection{Reporting and Recommendations Output}

\begin{itemize}
    \item Finalized alignment report approved by vCIO
    \item myITprocess updated with finalized risks and follow-up actions
\end{itemize}

% ------------------------------------------------------------
% Step 6
% ------------------------------------------------------------
\clearpage
\section{Follow-Up and Continuous Improvement}

\textbf{Goal:} Ensure recommendations are implemented and continuously improve the alignment process.

\subsection{Follow-Up Checklist}
Click \hyperref[sec:followup-checklist]{here} to see the full follow up checklist for full execution steps.
\begin{itemize}
    \item Confirm remediation completion via myITprocess
    \item Re-validate systems using Kaseya VSA
    \item Confirm backup success and restore capability (Datto)
    \item Verify security coverage remains active and compliant
    \item Update ITGlue documentation with final state
\end{itemize}

\subsection{Best Practices}

\begin{itemize}
    \item Schedule recurring alignment reviews in myITprocess
    \item Capture lessons learned to refine future alignment standards
    \item TAS ensures all technical actions are completed and documented
    \item vCIO reviews final outcomes to confirm business alignment and strategic impact
\end{itemize}

\subsection{Follow-Up and Continuous Improvement Output}

\begin{itemize}
    \item Completed remediation tracking
    \item Updated documentation and historical records
    \item Improved TAS workflow and alignment consistency
\end{itemize}
